\newpage
\section{Einleitung} \label{sec:Einleitung}
Text

\textbf{} 
%Thema und Zielsetzung: Stellen Sie zunächst Thema und Zielstellung der Arbeit vor.
%Theorie: Vermitteln Sie Ihre Theorie(n) über das Thema und geben Sie an, auf was sich Ihre Theorie stützt.
%Fragestellung: Teilen Sie mit, welche Fragen in der folgenden Arbeit beantwortet werden.
%Quellen: Welche Quellen haben Sie für Ihre Arbeit genutzt bzw. wie haben Sie Ihre Frage(n) beantwortet?
%Ergebnis: Führen Sie Ihre Ergebnisse auf, also teilen Sie mit, was Sie herausgefunden haben.
%Fazit: Stellen Sie am Ende des Abstracts eine Quintessenz auf. Sie können Ihr Fazit auch mit einer %Zukunftsprognose verbinden.

\subsection{Medizinische Kenntnisse - Pulsoxymetrie} 
Für die Messung des peripheren Pulses per Android-Uhr wird das Prinzip der Pulsosymetrie genutzt. \\
Dieses Verfahren benötigt zwei Sensoren: zum einen eine Lichtquelle, zum anderen ein Lichtsensor. Die Lichtquelle sendet Infrarot-Lichtwellen aus, die durch die Haut dringen. Der Sensor misst die Lichtanteile, die absorbiert wurden. \\
Die Lichtabsorption im Blut ist abhänig von der Hämoglobinkonzentration und der Sättigung des Hämoglobins mit Sauerstoff. Oxigeniertes und desoxigeniertes Hämoglobin schwächen das Licht jeweils charakteristisch ab. \\
Mit diesem Prinzip ist es auch möglich, die Sauerstoffsättigung im kapillären Blut gemessen werden.\\



% Kurze Erklärung 
% reicht das oder noch mehr?
% bitte Korrektur lesen
% hätte auch noch Bilder, die hier passen würden, allerdings weiß ich nicht, wie man die einfügt xD sorry!



