%% Basierend auf einer TeXnicCenter-Vorlage von Tino Weinkauf.
%%%%%%%%%%%%%%%%%%%%%%%%%%%%%%%%%%%%%%%%%%%%%%%%%%%%%%%%%%%%%%

%%%%%%%%%%%%%%%%%%%%%%%%%%%%%%%%%%%%%%%%%%%%%%%%%%%%%%%%%%%%%
%% HEADER
%%%%%%%%%%%%%%%%%%%%%%%%%%%%%%%%%%%%%%%%%%%%%%%%%%%%%%%%%%%%%
\documentclass[a4paper,oneside,12pt]{article}
\usepackage{geometry}
\geometry{a4paper,left=40mm,right=30mm, top=25mm, bottom=25mm} 
% Alternative Optionen:
%	Papiergr��e: a4paper / a5paper / b5paper / letterpaper / legalpaper / executivepaper
% Duplex: oneside / twoside
% Grundlegende Fontgr��en: 10pt / 11pt / 12pt


%% Deutsche Anpassungen %%%%%%%%%%%%%%%%%%%%%%%%%%%%%%%%%%%%%
\usepackage[ngerman]{babel}
\usepackage[utf8]{inputenc}
\usepackage[T1]{fontenc}
\usepackage{lmodern} %Type1-Schriftart f�r nicht-englische Texte
\usepackage{float}
\usepackage{floatflt}

%% Packages f�r Grafiken & Abbildungen %%%%%%%%%%%%%%%%%%%%%%
\usepackage{graphicx} %%Zum Laden von Grafiken
%\usepackage{subfig} %%Teilabbildungen in einer Abbildung
%\usepackage{pst-all} %%PSTricks - nicht verwendbar mit pdfLaTeX

%% Beachten Sie:
%% Die Einbindung einer Grafik erfolgt mit \includegraphics{Dateiname}
%% bzw. �ber den Dialog im Einf�gen-Men�.
%% 
%% Im Modus "LaTeX => PDF" k�nnen Sie u.a. folgende Grafikformate verwenden:
%%   .jpg  .png  .pdf  .mps
%% 
%% In den Modi "LaTeX => DVI", "LaTeX => PS" und "LaTeX => PS => PDF"
%% k�nnen Sie u.a. folgende Grafikformate verwenden:
%%   .eps  .ps  .bmp  .pict  .pntg


%% Packages f�r Formeln %%%%%%%%%%%%%%%%%%%%%%%%%%%%%%%%%%%%%
\usepackage{amsmath}
\usepackage{amsthm}
\usepackage{amsfonts}


%% Zeilenabstand %%%%%%%%%%%%%%%%%%%%%%%%%%%%%%%%%%%%%%%%%%%%
%\usepackage{setspace}
%\singlespacing        %% 1-zeilig (Standard)
%\onehalfspacing       %% 1,5-zeilig
%\doublespacing        %% 2-zeilig


%% Farben %%%%%%%%%%%%%%%%%%%%%%%%%%%%%%%%%%%%%%%%%%%%%%%%%%%
\usepackage{color}
\usepackage{framed}
\usepackage{colortbl}
\definecolor{middlegray}{rgb}{0.5,0.5,0.5}
\definecolor{lightgray}{rgb}{0.8,0.8,0.8}
\definecolor{orange}{rgb}{0.8,0.3,0.3}
\definecolor{yac}{rgb}{0.6,0.6,0.1}
\definecolor{darkgray}{rgb}{0.3,0.3,0.3}
\definecolor{blue}{rgb}{0,0,1}
\definecolor{green}{rgb}{0,0.6,0}
\definecolor{yellow}{rgb}{1,1,0}
\definecolor{shadecolor}{gray}{.85}
\usepackage[table]{xcolor}
\newcommand{\tblue}{\cellcolor{blue!25}}
\newcommand{\tred}{\cellcolor{red!25}}
\newcommand{\tgreen}{\cellcolor{green!25}}
\newcommand{\tyellow}{\cellcolor{yellow!25}}


%% CodeListings %%%%%%%%%%%%%%%%%%%%%%%%%%%%%%%%%%%%%%%%%%%%%
\usepackage{listings}
\lstset{
	basicstyle=\scriptsize\ttfamily,
	keywordstyle=\bfseries\ttfamily\color{blue},
	breaklines=true,
	stringstyle=\color{darkgray}\ttfamily,
	commentstyle=\color{green}\ttfamily,
	emph={square}, 
	emphstyle=\color{blue}\texttt,
	emph={[2]root,base},
	emphstyle={[2]\color{yac}\texttt},
	showstringspaces=false,
	flexiblecolumns=false,
	captionpos=b,
	tabsize=2,
	numbers=left,
	numberstyle=\tiny,
	numberblanklines=true,
	stepnumber=1,
	numbersep=10pt,
	xleftmargin=15pt,
	frame=single
}

%% Verlinktes Inhaltsverzeichnis %%%%%%%%%%%%%%%%%%%%%%%%%%%
\usepackage[colorlinks,
pdfpagelabels,
pdfstartview = FitH,
bookmarksopen = true,
bookmarksnumbered = true,
linkcolor = black,
plainpages = false,
hypertexnames = false,
citecolor = black] {hyperref}


%% Sonstiges %%%%%%%%%%%%%%%%%%%%%%%%%%%%%%%%%%%%%%%%%%%%%%%
\usepackage{float}
\usepackage{tabularx}
\usepackage{amsmath}
\usepackage{subfigure}
\usepackage{multirow}
\usepackage{ulem}
\usepackage{wrapfig}

\setlength{\parindent}{0pt}

%% Literaturverzeichnis
\usepackage{url}
\usepackage[backend=bibtex,block=none,style=numeric,sorting=none]{biblatex}
\addbibresource{literatur.bib}
\DeclareFieldFormat[misc]{title}{\textit{#1}\isdot}
\DeclareFieldFormat{url}{\newline\url{#1}}
\DefineBibliographyStrings{german}{%
  backrefpage = {Seite},% originally "cited on page"
  backrefpages = {Seiten},% originally "cited on pages"
}


%% Andere Packages %%%%%%%%%%%%%%%%%%%%%%%%%%%%%%%%%%%%%%%%%%
%\usepackage{a4wide} %%Kleinere Seitenr�nder = mehr Text pro Zeile.
%\usepackage{fancyhdr} %%Fancy Kopf- und Fu�zeilen
%\usepackage{longtable} %%F�r Tabellen, die eine Seite �berschreiten


%% execute scripts %%%%%%%%%%%%%%%%%%%%%%%%%%%%%%%%%%%%%%%%%%
\newcommand{\visioToPDF}[1] {
	\immediate\write18{convert_scripts\string\\visioToPDF.bat #1}
}


%%%%%%%%%%%%%%%%%%%%%%%%%%%%%%%%%%%%%%%%%%%%%%%%%%%%%%%%%%%%%
%% Anmerkungen
%%%%%%%%%%%%%%%%%%%%%%%%%%%%%%%%%%%%%%%%%%%%%%%%%%%%%%%%%%%%%
%
% Zu erledigen:
% 1. Passen Sie die Packages und deren Optionen an (siehe oben).
% 2. Wenn Sie wollen, erstellen Sie eine BibTeX-Datei
%    (z.B. 'literatur.bib').
% 3. Happy TeXing!
%
%%%%%%%%%%%%%%%%%%%%%%%%%%%%%%%%%%%%%%%%%%%%%%%%%%%%%%%%%%%%%


%%%%%%%%%%%%%%%%%%%%%%%%%%%%%%%%%%%%%%%%%%%%%%%%%%%%%%%%%%%%%
%% Optionen / Modifikationen
%%%%%%%%%%%%%%%%%%%%%%%%%%%%%%%%%%%%%%%%%%%%%%%%%%%%%%%%%%%%%

%\input{optionen} %Eine Datei 'optionen.tex' wird hierf�r ben�tigt.
%% ==> TeXnicCenter liefert m�gliche Optionendateien
%% ==> im Vorlagenarchiv mit (Datei | Neu von Vorlage...).


%%%%%%%%%%%%%%%%%%%%%%%%%%%%%%%%%%%%%%%%%%%%%%%%%%%%%%%%%%%%%
%% DOKUMENT
%%%%%%%%%%%%%%%%%%%%%%%%%%%%%%%%%%%%%%%%%%%%%%%%%%%%%%%%%%%%%
\begin{document}
\thispagestyle{empty}

\title {
	\huge \textsc{HeartRate2Go}
}
	
\author {
	\begin{tabular}{rl}
		\large Matthias Böffel & \small Matrikel Nr.: 864483 \\ 
		\large Patrick Mathias & \small Matrikel Nr.: 864089 \\ 
		\large Markus Nebel & \small Matrikel Nr.: 864681 \\ 
		\large Janina Sauer & \small Matrikel Nr.: 865235 \\ 
	\end{tabular}
}

\maketitle
\vfill
\begin{figure}[H]
\centering
\small Hochschule Kaiserslautern\\University of Applied Sciences\\
\bigskip
\large Betreuer: Prof. Dr.-Ing. Jan Conrad\\
\bigskip
\includegraphics[scale=0.4]{images/hskllogo.jpg}  
\end{figure}

\include{inhaltsverzeichnis}

\newpage
\section{Einleitung} \label{sec:Einleitung}
Text

\textbf{} 
%Thema und Zielsetzung: Stellen Sie zunächst Thema und Zielstellung der Arbeit vor.
%Theorie: Vermitteln Sie Ihre Theorie(n) über das Thema und geben Sie an, auf was sich Ihre Theorie stützt.
%Fragestellung: Teilen Sie mit, welche Fragen in der folgenden Arbeit beantwortet werden.
%Quellen: Welche Quellen haben Sie für Ihre Arbeit genutzt bzw. wie haben Sie Ihre Frage(n) beantwortet?
%Ergebnis: Führen Sie Ihre Ergebnisse auf, also teilen Sie mit, was Sie herausgefunden haben.
%Fazit: Stellen Sie am Ende des Abstracts eine Quintessenz auf. Sie können Ihr Fazit auch mit einer %Zukunftsprognose verbinden.

\subsection{Medizinische Kenntnisse - Pulsoxymetrie}

% Kurze Erklärung 




\newpage
\section{Android} \label{sec:hauptteil_android}

\subsection{Android (Standard)}
Das Android-Betriebsystem erfreut sich weltweit größter Beliebtheit. Android ist mit über 
75\% auf dem Markt das meist verbreitete Mobil-Betriebsystem für Smartphones und Tablets. 
Ein großer Vorteil des mobilen Betriebssystems ist die Möglichkeit, die Funktionalität 
durch Installation zusätzlicher Anwendungen (Apps) zu erweitern. Android liegt aktuell
in der Version 5.0 (Lollipop) vor, wobei die vorliegende Implementierung der Version
4.4 zugrunde liegt.
\\[0.5cm]
Erstellt werden Apps i.d.R. mit Hilfe des Android Frameworks in der Programmiersprache Java. 
Das Android Framework passt in die Kategorie der modernen Gui-Frameworks, da die Programmlogik
strikt von den Definitionen für das Layout getrennt ist. Das Layout wird durch Dateien mit
XML-Struktur festegelegt und kann so unabhängig vom Code angepasst werden.

\subsection{Android Wear}
Android Wear als noch recht neues Bestriebssystem stellt eine ressourcenschonende Version
des Standard-Android-Betriebssystems für Wearables (Smartwatches, Armbänder, etc.) dar. 
Die Wearables bringen in den meisten Fällen Sensoren für Fitness-Tracking (z.B. Pulsmesser,
und Schrittzähler) mit, die durch die Android API bereits unterstützt werden. Das Wearable-Gerät 
kann zwar selbstständig agieren, ist jedoch ohne entsprechende Hardware zur Nutzung von Internet,
W-Lan oder anderen Ressourcen auf ein gekoppeltes Handheld-Gerät angewiesen. Auch Hersteller
Google betont, dass das Betriebsystem grundlegend zur Kopplung mit einem Smartphone bzw. Tablet
(Im folgenden allgemein: Handheld) ausgelegt ist. Notifications vom Handheld, werden beispielsweise
bequem auf das Wearable-Gerät weitergeleitet, während in die andere Richtung Spracheingaben auf dem
Wearable-Gerät interpretiert und zum Handheld-Gerät zur weiteren Verarbeitung übermittelt werden können.
Die Kommunikation findet dabei i.d.R. über eine spezielle Wearable-Bluetooth-API statt.
\\[0.5cm]
Die Design-Prinzipien, die grundlegend auf den Entwicklerseiten von Android Wear empfohlen werden,
unterstreichen ebenfalls die enge Verbundenheit zum Handheld-Gerät. So sollen rechen- bzw. zeitintensive
Tasks auf das leistungsfähigere Handheld-gerät ausgelagert werden und Konfigurationen für
Wearable-Apps weitesgehend auf dem Handheld-Gerät vorgenommen werden. So bietet es sich an,
für eine Wearable-App gleich eine zugehörige Handheld-App mitzuliefern. Die Installation einer
Wearable-App erfolgt dabei auch über das Handheld-Gerät: Eine APK-Installations-Datei kann mehrere
Apps für verschiedene Geräte enthalten, die dann automatisch verteilt werden. Apps für Android Wear werden unter den gleichen Bedingungen erstellt, wie Apps für das Standard-Android-Betriebssystem. Zusätzlich unterstützt Android Wear sowohl runde und quadratische Display-Typen, was bei der Gestaltung des Layouts zu beachten ist.

%\bigskip
%\begin{figure}[H]
	%\visioToPDF{images/datei.pdf}
	%\centering
	%\includegraphics[scale=0.7]{images/datei.pdf}
	%\caption{Beschriftung}
	%\label{fig:layer}
%\end{figure}
%\bigskip

%\lstset{language=Java}
%\begin{lstlisting}[caption=Listing, label=lst:Listing]
%\end{lstlisting}

%\begin{shaded}
%Text in Textbox
%\end{shaded}

%\ref{sec:hauptteil}
%\cite{audio_architecture}
\newpage
\section{Hauptteil - Qt} \label{sec:hauptteil_qt}
\subsection{Hauptteil - Part1}
Text

%\bigskip
%\begin{figure}[H]
	%\visioToPDF{images/datei.pdf}
	%\centering
	%\includegraphics[scale=0.7]{images/datei.pdf}
	%\caption{Beschriftung}
	%\label{fig:layer}
%\end{figure}
%\bigskip

%\lstset{language=Java}
%\begin{lstlisting}[caption=Listing, label=lst:Listing]
%\end{lstlisting}

%\begin{shaded}
%Text in Textbox
%\end{shaded}

%\ref{sec:hauptteil}
%\cite{audio_architecture}
\newpage
\section{Fazit}
Text
% muss rein (von Conrad):
% implementierte Anwendung und deren Funktionalitäten
% Verwendete Frameworks mit Begründung
% Beschreibung des Software-Designs
% Beschreibung des User Interfaces
% kurze Beschreibung der Bedienung der Anwendung
% weitere sinnvolle Angaben
% Verwendete Literatur und Quellen
% Retrospektive (kritischer Rückblick, Vergleich zu ursprüngliche Planung)
% - Schwierigkeit bei der Entscheidung, was schaffen wir in der vorgegen Zeit?
% - Übertragung von Samrtphone zur GUI nicht über Bluetooth möglich 
% Ausblick (Wie kann das Projekt weiter entwickelt werden)
% - Messung per Smartphone
% - siehe Konzeptpaper (fehlende Nice-to-have Lasten)
% 	- Mehrsprachigkeit (Deutsch und Englisch)
% 	- Benutzerprofile anlegen und abspeichern mit Anamneseabfrage
% 	- Berechnung des Kalorienverbrauchs
% - Marker setzten

% Drucken und mit abgeben:
% Ausgabe der Projektarbeit und Nutzungsvereinbarung
% Ehrenwörtliche Erklärung

\include{verzeichnisse}
\end{document}

