\newpage
\section{Android} \label{sec:hauptteil_android}

\subsection{Android (Standard)}
Das Android-Betriebsystem erfreut sich weltweit größter Beliebtheit. Android ist mit über 
75\% auf dem Markt das meist verbreitete Mobil-Betriebsystem für Smartphones und Tablets. 
Ein großer Vorteil des mobilen Betriebssystems ist die Möglichkeit, die Funktionalität 
durch Installation zusätzlicher Anwendungen (Apps) zu erweitern. Android liegt aktuell
in der Version 5.0 (Lollipop) vor, wobei die vorliegende Implementierung der Version
4.4 zugrunde liegt.
\\[0.5cm]
Erstellt werden Apps i.d.R. mit Hilfe des Android Frameworks in der Programmiersprache Java. 
Das Android Framework passt in die Kategorie der modernen Gui-Frameworks, da die Programmlogik
strikt von den Definitionen für das Layout getrennt ist. Das Layout wird durch Dateien mit
XML-Struktur festegelegt und kann so unabhängig vom Code angepasst werden.

\subsection{Android Wear}
Android Wear als noch recht neues Bestriebssystem stellt eine ressourcenschonende Version
des Standard-Android-Betriebssystems für Wearables (Smartwatches, Armbänder, etc.) dar. 
Die Wearables bringen in den meisten Fällen Sensoren für Fitness-Tracking (z.B. Pulsmesser,
und Schrittzähler) mit, die durch die Android API bereits unterstützt werden. Das Wearable-Gerät 
kann zwar selbstständig agieren, ist jedoch ohne entsprechende Hardware zur Nutzung von Internet,
W-Lan oder anderen Ressourcen auf ein gekoppeltes Handheld-Gerät angewiesen. Auch Hersteller
Google betont, dass das Betriebsystem grundlegend zur Kopplung mit einem Smartphone bzw. Tablet
(Im folgenden allgemein: Handheld) ausgelegt ist. Notifications vom Handheld, werden beispielsweise
bequem auf das Wearable-Gerät weitergeleitet, während in die andere Richtung Spracheingaben auf dem
Wearable-Gerät interpretiert und zum Handheld-Gerät zur weiteren Verarbeitung übermittelt werden können.
Die Kommunikation findet dabei i.d.R. über eine spezielle Wearable-Bluetooth-API statt.
\\[0.5cm]
Die Design-Prinzipien, die grundlegend auf den Entwicklerseiten von Android Wear empfohlen werden,
unterstreichen ebenfalls die enge Verbundenheit zum Handheld-Gerät. So sollen rechen- bzw. zeitintensive
Tasks auf das leistungsfähigere Handheld-gerät ausgelagert werden und Konfigurationen für
Wearable-Apps weitesgehend auf dem Handheld-Gerät vorgenommen werden. So bietet es sich an,
für eine Wearable-App gleich eine zugehörige Handheld-App mitzuliefern. Die Installation einer
Wearable-App erfolgt dabei auch über das Handheld-Gerät: Eine APK-Installations-Datei kann mehrere
Apps für verschiedene Geräte enthalten, die dann automatisch verteilt werden. Apps für Android Wear werden unter den gleichen Bedingungen erstellt, wie Apps für das Standard-Android-Betriebssystem. Zusätzlich unterstützt Android Wear sowohl runde und quadratische Display-Typen, was bei der Gestaltung des Layouts zu beachten ist.

%\bigskip
%\begin{figure}[H]
	%\visioToPDF{images/datei.pdf}
	%\centering
	%\includegraphics[scale=0.7]{images/datei.pdf}
	%\caption{Beschriftung}
	%\label{fig:layer}
%\end{figure}
%\bigskip

%\lstset{language=Java}
%\begin{lstlisting}[caption=Listing, label=lst:Listing]
%\end{lstlisting}

%\begin{shaded}
%Text in Textbox
%\end{shaded}

%\ref{sec:hauptteil}
%\cite{audio_architecture}