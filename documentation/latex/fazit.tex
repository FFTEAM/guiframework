\newpage
\section{Fazit}

Das Ziel dieses Kapitels ist eine kritische Stellungnahme bezüglich des Projektverlaufs. Hierfür wird ein Vergleich zwischen den geplanten und erreichten Zielen durchgeführt. Des Weiteren ist es von Vorteil, die positiven sowie negativen Aspekte, die während des Projektverlauf aufgetaucht sind, ausführlich zu diskutieren. Zum Abschluss des Kapitels werden gezielte Erweiterungsmöglichkeiten in Bezug auf spätere Projekte diskutiert.

\subsection{Retrospektive} \label{sec:Retrospektive}

Bei der Umsetzung des geplanten Projektes traten nur wenige Schwierigkeiten auf. Die Arbeitsgruppe konnte durch Internet-Messaging zeitunabhängig kommunizieren und Gedanken, Anregungen, Kritik sowie Ideen austauschen. Die Projektdateien unterlagen vollständig einer Online-Versionskontrolle, um gemeinsame Zugriffe zu koordinieren und Änderungen angemessen protokollieren zu können. Durch die Trennung von Design und Programmlogik innerhalb der beiden verwendeten Frameworks konnten die Zuständigkeiten von Beginn an klar geregelt werden (siehe Tabelle \ref{tbl:zustaendigkeiten}).\\
\begin{table*}[h]
	\centering
		\begin{tabularx}{\textwidth}{l|X}
			\hline
			Teilnehmer & Arbeitsaufgabe \\
			\hline
			\hline
			Matthias Böffel & Android Programmlogik\\
			\hline
			Patrick Mathias & Qt Design\\
			\hline
			Markus Nebel & Qt Programmlogik\\
			\hline
			Janina Sauer & Android Design, medizinisches Wissen\\
			\hline
		\end{tabularx}
		\caption{Arbeitsaufgaben der Teilnehmer}
		\label{tbl:zustaendigkeiten}
\end{table*}
\\
Es konnten alle im Konzeptpapier geforderten Grundfunktionalitäten implementiert werden. Darüber hinaus wurden die meisten optionalen Features umgesetzt. Lediglich die Bewertung der Messdaten durch Benutzerprofile mit Anamnese-Werten und die Berechnung des Kalorienverbrauchs flossen aufgrund mangelnder Zeit nicht mehr in das Projekt mit ein.Außerdem musste die TCP-Übertragung als Alternative zur ausgehenden Bluetooth-Verbindung verwendet werden.
\\[0.5cm]
Das QT-Framework und das Android Framework wurden von den Teilnehmern bereits in anderen Projekten verwendet. Die QML-Komponente und das Android Wear Framework waren allerdings bis dahin unbekannt und erforderten Einarbeitung in die Materie. In den jeweiligen Abschnitten (Android: Abschnitt \ref{sec:einschraenkungen} / QT: Abschnitt \ref{sec:Probleme bezüglich QT}) werden weitere Probleme bzw. komplexe Punkte aufgezeigt.
% Retrospektive (kritischer Rückblick, Vergleich zu ursprüngliche Planung)
% - Schwierigkeit bei der Entscheidung, was schaffen wir in der vorgegen Zeit?
% - Übertragung von Smartphone zur GUI nicht über Bluetooth möglich 
% Ausblick (Wie kann das Projekt weiter entwickelt werden)
\subsection{Ausblick} \label{sec:Ausblick}
Wie schon im Konzeptpapier erwähnt, besteht die Möglichkeit, das Projekt \textit{HeartRate2Go} zu publizieren und es so anderen Anwendern zugänglich zu machen. Hierzu könnte es durch weitere Funktionen erweitert werden. Einige dieser Anwendungen finden sich schon im Konzeptpapier unter 2.b. Optionale Funktionen, zum Beispiel: das Anlegen von Benutzerprofilen. \\ Dies geschieht derzeit nur ansatzweise, die gesendeten Werte werden für jedes Benutzerprofil des Betriebssystems separat abgespeichert. Jedoch ist eine Anamneseabfrage noch nicht möglich. In dieser würde nach Alter, Geschlecht, Größe, Gewicht, maximaler und minimaler Pulswert für die beiden Messwerttypen gefragt werden. So wäre auch eine erste Einschätzung der gemessenen Werte möglich.
\\
Ein anderer Punkt, ist die Berechnung des Kalorienverbrauchs. Zwar wird während einer Aktivitätsmessung die Anzahl der Schritte angezeigt, jedoch war es leider in der vorgegeben Zeit nicht möglich, der dadurch resultierende Kalorienverbrauch zu errechnen. Hierfür ist auch die Schrittlänge nötig, die mit einem Benutzerprofil einhergeht. \\
Des Weiteren stand zur Diskussion, ob dem Nutzer die Möglichkeit gegeben werden soll, Marker zusetzen, die eine besondere Situation kennzeichnen und in der späteren Ansicht speziell angezeigt werden. Dies ist bei einer, vom Hausarzt angeordneten, Langzeit-EKG-Messung ein wichtiger Teil, auch für die spätere Bewertung der Messung. \\
Da das \textit{HeartRate2Go-Programm} auf allen Betriebssysteme läuft, wäre eine App für das iOS-Betriebssystem auch praktisch. Der Zeit existiert die \textit{HeartRate2Go-App} nur für Android. Die Erstellung einer iOS-App war jedoch leider nicht möglich, da hierfür keine passende Apple-Komponenten zur Verfügung standen.\\
Die Umsetzung der genannten Punkte scheiterte an der begrenzten Zeit, die für dieses Projekt zur Verfügung stand. \\
% - Messung per Smartphone
% - siehe Konzeptpaper (fehlende Nice-to-have Lasten)
% 	- Benutzerprofile anlegen und abspeichern mit Anamneseabfrage
% 	- Berechnung des Kalorienverbrauchs
% - Marker setzten
% - iOS-App

% Drucken und mit abgeben:
% Ausgabe der Projektarbeit und Nutzungsvereinbarung
% Ehrenwörtliche Erklärung