


\newpage
\section{Fazit}
% muss rein (von Conrad):
% implementierte Anwendung und deren Funktionalitäten
% Verwendete Frameworks mit Begründung
% Beschreibung des Software-Designs
% Beschreibung des User Interfaces
% kurze Beschreibung der Bedienung der Anwendung
% weitere sinnvolle Angaben
% Verwendete Literatur und Quellen
\subsection{Retrospektive} \label{sec:Retrospektive}

% Retrospektive (kritischer Rückblick, Vergleich zu ursprüngliche Planung)
% - Schwierigkeit bei der Entscheidung, was schaffen wir in der vorgegen Zeit?
% - Übertragung von Smartphone zur GUI nicht über Bluetooth möglich 
% - Print Dialog
% Ausblick (Wie kann das Projekt weiter entwickelt werden)
\subsection{Ausblick} \label{sec:Ausblick}
Wie schon im Konzeptpapier erwähnt, besteht die Möglichkeit, das Projekt \textit{HeartRate2Go} zu publizieren und es so anderen Anwendern zugänglich zu machen. Hierzu könnte es durch weitere Funktionen erweitert werden. Einige dieser Anwendungen finden sich schon im Konzeptpapier unter 2.b. Optionale Funktionen, zum Beispiel: das Anlegen von Benutzerprofilen. \\ Dies geschieht derzeit nur ansatzweise, die gesendeten Werte werden für jedes Benutzerprofil des Betriebssystems separat abgespeichert. Jedoch ist eine Anamneseabfrage noch nicht möglich. In dieser würde nach Alter, Geschlecht, Größe, Gewicht, maximaler und minimaler Pulswert für die beiden Messwerttypen gefragt werden. So wäre auch eine erste Einschätzung der gemessenen Werte möglich.
\\
Ein anderer Punkt, ist die Berechnung des Kalorienverbrauchs. Zwar wird während einer Aktivitätsmessung die Anzahl der Schritte angezeigt, jedoch war es leider in der vorgegeben Zeit nicht möglich, der dadurch resultierende Kalorienverbrauch zu errechnen. Hierfür ist auch die Schrittlänge nötig, die mit einem Benutzerprofil einhergeht. \\
Des Weiteren stand zur Diskussion, ob dem Nutzer die Möglichkeit gegeben werden soll, Marker zusetzen, die eine besondere Situation kennzeichnen und in der späteren Ansicht speziell angezeigt werden. Dies ist bei einer, vom Hausarzt angeordneten, Langzeit-EKG-Messung ein wichtiger Teil, auch für die spätere Bewertung der Messung. \\
Da das \textit{HeartRate2Go-Programm} auf allen Betriebssysteme läuft, wäre eine App für das iOS-Betriebssystem auch praktisch. Der Zeit existiert die \textit{HeartRate2Go-App} nur für Android. Die Erstellung einer iOS-App war jedoch leider nicht möglich, da hierfür keine passende Apple-Komponenten zur Verfügung standen.\\
Die Umsetzung der genannten Punkte scheiterte an der begrenzten Zeit, die für dieses Projekt zur Verfügung stand. \\
% - Messung per Smartphone
% - siehe Konzeptpaper (fehlende Nice-to-have Lasten)
% 	- Benutzerprofile anlegen und abspeichern mit Anamneseabfrage
% 	- Berechnung des Kalorienverbrauchs
% - Marker setzten
% - iOS-App

% Drucken und mit abgeben:
% Ausgabe der Projektarbeit und Nutzungsvereinbarung
% Ehrenwörtliche Erklärung